
% Default to the notebook output style

    


% Inherit from the specified cell style.




    
\documentclass[11pt]{article}

    
    
    \usepackage[T1]{fontenc}
    % Nicer default font (+ math font) than Computer Modern for most use cases
    \usepackage{mathpazo}

    % Basic figure setup, for now with no caption control since it's done
    % automatically by Pandoc (which extracts ![](path) syntax from Markdown).
    \usepackage{graphicx}
    % We will generate all images so they have a width \maxwidth. This means
    % that they will get their normal width if they fit onto the page, but
    % are scaled down if they would overflow the margins.
    \makeatletter
    \def\maxwidth{\ifdim\Gin@nat@width>\linewidth\linewidth
    \else\Gin@nat@width\fi}
    \makeatother
    \let\Oldincludegraphics\includegraphics
    % Set max figure width to be 80% of text width, for now hardcoded.
    \renewcommand{\includegraphics}[1]{\Oldincludegraphics[width=.8\maxwidth]{#1}}
    % Ensure that by default, figures have no caption (until we provide a
    % proper Figure object with a Caption API and a way to capture that
    % in the conversion process - todo).
    \usepackage{caption}
    \DeclareCaptionLabelFormat{nolabel}{}
    \captionsetup{labelformat=nolabel}

    \usepackage{adjustbox} % Used to constrain images to a maximum size 
    \usepackage{xcolor} % Allow colors to be defined
    \usepackage{enumerate} % Needed for markdown enumerations to work
    \usepackage{geometry} % Used to adjust the document margins
    \usepackage{amsmath} % Equations
    \usepackage{amssymb} % Equations
    \usepackage{textcomp} % defines textquotesingle
    % Hack from http://tex.stackexchange.com/a/47451/13684:
    \AtBeginDocument{%
        \def\PYZsq{\textquotesingle}% Upright quotes in Pygmentized code
    }
    \usepackage{upquote} % Upright quotes for verbatim code
    \usepackage{eurosym} % defines \euro
    \usepackage[mathletters]{ucs} % Extended unicode (utf-8) support
    \usepackage[utf8x]{inputenc} % Allow utf-8 characters in the tex document
    \usepackage{fancyvrb} % verbatim replacement that allows latex
    \usepackage{grffile} % extends the file name processing of package graphics 
                         % to support a larger range 
    % The hyperref package gives us a pdf with properly built
    % internal navigation ('pdf bookmarks' for the table of contents,
    % internal cross-reference links, web links for URLs, etc.)
    \usepackage{hyperref}
    \usepackage{longtable} % longtable support required by pandoc >1.10
    \usepackage{booktabs}  % table support for pandoc > 1.12.2
    \usepackage[inline]{enumitem} % IRkernel/repr support (it uses the enumerate* environment)
    \usepackage[normalem]{ulem} % ulem is needed to support strikethroughs (\sout)
                                % normalem makes italics be italics, not underlines
    

    
    
    % Colors for the hyperref package
    \definecolor{urlcolor}{rgb}{0,.145,.698}
    \definecolor{linkcolor}{rgb}{.71,0.21,0.01}
    \definecolor{citecolor}{rgb}{.12,.54,.11}

    % ANSI colors
    \definecolor{ansi-black}{HTML}{3E424D}
    \definecolor{ansi-black-intense}{HTML}{282C36}
    \definecolor{ansi-red}{HTML}{E75C58}
    \definecolor{ansi-red-intense}{HTML}{B22B31}
    \definecolor{ansi-green}{HTML}{00A250}
    \definecolor{ansi-green-intense}{HTML}{007427}
    \definecolor{ansi-yellow}{HTML}{DDB62B}
    \definecolor{ansi-yellow-intense}{HTML}{B27D12}
    \definecolor{ansi-blue}{HTML}{208FFB}
    \definecolor{ansi-blue-intense}{HTML}{0065CA}
    \definecolor{ansi-magenta}{HTML}{D160C4}
    \definecolor{ansi-magenta-intense}{HTML}{A03196}
    \definecolor{ansi-cyan}{HTML}{60C6C8}
    \definecolor{ansi-cyan-intense}{HTML}{258F8F}
    \definecolor{ansi-white}{HTML}{C5C1B4}
    \definecolor{ansi-white-intense}{HTML}{A1A6B2}

    % commands and environments needed by pandoc snippets
    % extracted from the output of `pandoc -s`
    \providecommand{\tightlist}{%
      \setlength{\itemsep}{0pt}\setlength{\parskip}{0pt}}
    \DefineVerbatimEnvironment{Highlighting}{Verbatim}{commandchars=\\\{\}}
    % Add ',fontsize=\small' for more characters per line
    \newenvironment{Shaded}{}{}
    \newcommand{\KeywordTok}[1]{\textcolor[rgb]{0.00,0.44,0.13}{\textbf{{#1}}}}
    \newcommand{\DataTypeTok}[1]{\textcolor[rgb]{0.56,0.13,0.00}{{#1}}}
    \newcommand{\DecValTok}[1]{\textcolor[rgb]{0.25,0.63,0.44}{{#1}}}
    \newcommand{\BaseNTok}[1]{\textcolor[rgb]{0.25,0.63,0.44}{{#1}}}
    \newcommand{\FloatTok}[1]{\textcolor[rgb]{0.25,0.63,0.44}{{#1}}}
    \newcommand{\CharTok}[1]{\textcolor[rgb]{0.25,0.44,0.63}{{#1}}}
    \newcommand{\StringTok}[1]{\textcolor[rgb]{0.25,0.44,0.63}{{#1}}}
    \newcommand{\CommentTok}[1]{\textcolor[rgb]{0.38,0.63,0.69}{\textit{{#1}}}}
    \newcommand{\OtherTok}[1]{\textcolor[rgb]{0.00,0.44,0.13}{{#1}}}
    \newcommand{\AlertTok}[1]{\textcolor[rgb]{1.00,0.00,0.00}{\textbf{{#1}}}}
    \newcommand{\FunctionTok}[1]{\textcolor[rgb]{0.02,0.16,0.49}{{#1}}}
    \newcommand{\RegionMarkerTok}[1]{{#1}}
    \newcommand{\ErrorTok}[1]{\textcolor[rgb]{1.00,0.00,0.00}{\textbf{{#1}}}}
    \newcommand{\NormalTok}[1]{{#1}}
    
    % Additional commands for more recent versions of Pandoc
    \newcommand{\ConstantTok}[1]{\textcolor[rgb]{0.53,0.00,0.00}{{#1}}}
    \newcommand{\SpecialCharTok}[1]{\textcolor[rgb]{0.25,0.44,0.63}{{#1}}}
    \newcommand{\VerbatimStringTok}[1]{\textcolor[rgb]{0.25,0.44,0.63}{{#1}}}
    \newcommand{\SpecialStringTok}[1]{\textcolor[rgb]{0.73,0.40,0.53}{{#1}}}
    \newcommand{\ImportTok}[1]{{#1}}
    \newcommand{\DocumentationTok}[1]{\textcolor[rgb]{0.73,0.13,0.13}{\textit{{#1}}}}
    \newcommand{\AnnotationTok}[1]{\textcolor[rgb]{0.38,0.63,0.69}{\textbf{\textit{{#1}}}}}
    \newcommand{\CommentVarTok}[1]{\textcolor[rgb]{0.38,0.63,0.69}{\textbf{\textit{{#1}}}}}
    \newcommand{\VariableTok}[1]{\textcolor[rgb]{0.10,0.09,0.49}{{#1}}}
    \newcommand{\ControlFlowTok}[1]{\textcolor[rgb]{0.00,0.44,0.13}{\textbf{{#1}}}}
    \newcommand{\OperatorTok}[1]{\textcolor[rgb]{0.40,0.40,0.40}{{#1}}}
    \newcommand{\BuiltInTok}[1]{{#1}}
    \newcommand{\ExtensionTok}[1]{{#1}}
    \newcommand{\PreprocessorTok}[1]{\textcolor[rgb]{0.74,0.48,0.00}{{#1}}}
    \newcommand{\AttributeTok}[1]{\textcolor[rgb]{0.49,0.56,0.16}{{#1}}}
    \newcommand{\InformationTok}[1]{\textcolor[rgb]{0.38,0.63,0.69}{\textbf{\textit{{#1}}}}}
    \newcommand{\WarningTok}[1]{\textcolor[rgb]{0.38,0.63,0.69}{\textbf{\textit{{#1}}}}}
    
    
    % Define a nice break command that doesn't care if a line doesn't already
    % exist.
    \def\br{\hspace*{\fill} \\* }
    % Math Jax compatability definitions
    \def\gt{>}
    \def\lt{<}
    % Document parameters
    \title{r\_basics}
    
    
    

    % Pygments definitions
    
\makeatletter
\def\PY@reset{\let\PY@it=\relax \let\PY@bf=\relax%
    \let\PY@ul=\relax \let\PY@tc=\relax%
    \let\PY@bc=\relax \let\PY@ff=\relax}
\def\PY@tok#1{\csname PY@tok@#1\endcsname}
\def\PY@toks#1+{\ifx\relax#1\empty\else%
    \PY@tok{#1}\expandafter\PY@toks\fi}
\def\PY@do#1{\PY@bc{\PY@tc{\PY@ul{%
    \PY@it{\PY@bf{\PY@ff{#1}}}}}}}
\def\PY#1#2{\PY@reset\PY@toks#1+\relax+\PY@do{#2}}

\expandafter\def\csname PY@tok@w\endcsname{\def\PY@tc##1{\textcolor[rgb]{0.73,0.73,0.73}{##1}}}
\expandafter\def\csname PY@tok@c\endcsname{\let\PY@it=\textit\def\PY@tc##1{\textcolor[rgb]{0.25,0.50,0.50}{##1}}}
\expandafter\def\csname PY@tok@cp\endcsname{\def\PY@tc##1{\textcolor[rgb]{0.74,0.48,0.00}{##1}}}
\expandafter\def\csname PY@tok@k\endcsname{\let\PY@bf=\textbf\def\PY@tc##1{\textcolor[rgb]{0.00,0.50,0.00}{##1}}}
\expandafter\def\csname PY@tok@kp\endcsname{\def\PY@tc##1{\textcolor[rgb]{0.00,0.50,0.00}{##1}}}
\expandafter\def\csname PY@tok@kt\endcsname{\def\PY@tc##1{\textcolor[rgb]{0.69,0.00,0.25}{##1}}}
\expandafter\def\csname PY@tok@o\endcsname{\def\PY@tc##1{\textcolor[rgb]{0.40,0.40,0.40}{##1}}}
\expandafter\def\csname PY@tok@ow\endcsname{\let\PY@bf=\textbf\def\PY@tc##1{\textcolor[rgb]{0.67,0.13,1.00}{##1}}}
\expandafter\def\csname PY@tok@nb\endcsname{\def\PY@tc##1{\textcolor[rgb]{0.00,0.50,0.00}{##1}}}
\expandafter\def\csname PY@tok@nf\endcsname{\def\PY@tc##1{\textcolor[rgb]{0.00,0.00,1.00}{##1}}}
\expandafter\def\csname PY@tok@nc\endcsname{\let\PY@bf=\textbf\def\PY@tc##1{\textcolor[rgb]{0.00,0.00,1.00}{##1}}}
\expandafter\def\csname PY@tok@nn\endcsname{\let\PY@bf=\textbf\def\PY@tc##1{\textcolor[rgb]{0.00,0.00,1.00}{##1}}}
\expandafter\def\csname PY@tok@ne\endcsname{\let\PY@bf=\textbf\def\PY@tc##1{\textcolor[rgb]{0.82,0.25,0.23}{##1}}}
\expandafter\def\csname PY@tok@nv\endcsname{\def\PY@tc##1{\textcolor[rgb]{0.10,0.09,0.49}{##1}}}
\expandafter\def\csname PY@tok@no\endcsname{\def\PY@tc##1{\textcolor[rgb]{0.53,0.00,0.00}{##1}}}
\expandafter\def\csname PY@tok@nl\endcsname{\def\PY@tc##1{\textcolor[rgb]{0.63,0.63,0.00}{##1}}}
\expandafter\def\csname PY@tok@ni\endcsname{\let\PY@bf=\textbf\def\PY@tc##1{\textcolor[rgb]{0.60,0.60,0.60}{##1}}}
\expandafter\def\csname PY@tok@na\endcsname{\def\PY@tc##1{\textcolor[rgb]{0.49,0.56,0.16}{##1}}}
\expandafter\def\csname PY@tok@nt\endcsname{\let\PY@bf=\textbf\def\PY@tc##1{\textcolor[rgb]{0.00,0.50,0.00}{##1}}}
\expandafter\def\csname PY@tok@nd\endcsname{\def\PY@tc##1{\textcolor[rgb]{0.67,0.13,1.00}{##1}}}
\expandafter\def\csname PY@tok@s\endcsname{\def\PY@tc##1{\textcolor[rgb]{0.73,0.13,0.13}{##1}}}
\expandafter\def\csname PY@tok@sd\endcsname{\let\PY@it=\textit\def\PY@tc##1{\textcolor[rgb]{0.73,0.13,0.13}{##1}}}
\expandafter\def\csname PY@tok@si\endcsname{\let\PY@bf=\textbf\def\PY@tc##1{\textcolor[rgb]{0.73,0.40,0.53}{##1}}}
\expandafter\def\csname PY@tok@se\endcsname{\let\PY@bf=\textbf\def\PY@tc##1{\textcolor[rgb]{0.73,0.40,0.13}{##1}}}
\expandafter\def\csname PY@tok@sr\endcsname{\def\PY@tc##1{\textcolor[rgb]{0.73,0.40,0.53}{##1}}}
\expandafter\def\csname PY@tok@ss\endcsname{\def\PY@tc##1{\textcolor[rgb]{0.10,0.09,0.49}{##1}}}
\expandafter\def\csname PY@tok@sx\endcsname{\def\PY@tc##1{\textcolor[rgb]{0.00,0.50,0.00}{##1}}}
\expandafter\def\csname PY@tok@m\endcsname{\def\PY@tc##1{\textcolor[rgb]{0.40,0.40,0.40}{##1}}}
\expandafter\def\csname PY@tok@gh\endcsname{\let\PY@bf=\textbf\def\PY@tc##1{\textcolor[rgb]{0.00,0.00,0.50}{##1}}}
\expandafter\def\csname PY@tok@gu\endcsname{\let\PY@bf=\textbf\def\PY@tc##1{\textcolor[rgb]{0.50,0.00,0.50}{##1}}}
\expandafter\def\csname PY@tok@gd\endcsname{\def\PY@tc##1{\textcolor[rgb]{0.63,0.00,0.00}{##1}}}
\expandafter\def\csname PY@tok@gi\endcsname{\def\PY@tc##1{\textcolor[rgb]{0.00,0.63,0.00}{##1}}}
\expandafter\def\csname PY@tok@gr\endcsname{\def\PY@tc##1{\textcolor[rgb]{1.00,0.00,0.00}{##1}}}
\expandafter\def\csname PY@tok@ge\endcsname{\let\PY@it=\textit}
\expandafter\def\csname PY@tok@gs\endcsname{\let\PY@bf=\textbf}
\expandafter\def\csname PY@tok@gp\endcsname{\let\PY@bf=\textbf\def\PY@tc##1{\textcolor[rgb]{0.00,0.00,0.50}{##1}}}
\expandafter\def\csname PY@tok@go\endcsname{\def\PY@tc##1{\textcolor[rgb]{0.53,0.53,0.53}{##1}}}
\expandafter\def\csname PY@tok@gt\endcsname{\def\PY@tc##1{\textcolor[rgb]{0.00,0.27,0.87}{##1}}}
\expandafter\def\csname PY@tok@err\endcsname{\def\PY@bc##1{\setlength{\fboxsep}{0pt}\fcolorbox[rgb]{1.00,0.00,0.00}{1,1,1}{\strut ##1}}}
\expandafter\def\csname PY@tok@kc\endcsname{\let\PY@bf=\textbf\def\PY@tc##1{\textcolor[rgb]{0.00,0.50,0.00}{##1}}}
\expandafter\def\csname PY@tok@kd\endcsname{\let\PY@bf=\textbf\def\PY@tc##1{\textcolor[rgb]{0.00,0.50,0.00}{##1}}}
\expandafter\def\csname PY@tok@kn\endcsname{\let\PY@bf=\textbf\def\PY@tc##1{\textcolor[rgb]{0.00,0.50,0.00}{##1}}}
\expandafter\def\csname PY@tok@kr\endcsname{\let\PY@bf=\textbf\def\PY@tc##1{\textcolor[rgb]{0.00,0.50,0.00}{##1}}}
\expandafter\def\csname PY@tok@bp\endcsname{\def\PY@tc##1{\textcolor[rgb]{0.00,0.50,0.00}{##1}}}
\expandafter\def\csname PY@tok@fm\endcsname{\def\PY@tc##1{\textcolor[rgb]{0.00,0.00,1.00}{##1}}}
\expandafter\def\csname PY@tok@vc\endcsname{\def\PY@tc##1{\textcolor[rgb]{0.10,0.09,0.49}{##1}}}
\expandafter\def\csname PY@tok@vg\endcsname{\def\PY@tc##1{\textcolor[rgb]{0.10,0.09,0.49}{##1}}}
\expandafter\def\csname PY@tok@vi\endcsname{\def\PY@tc##1{\textcolor[rgb]{0.10,0.09,0.49}{##1}}}
\expandafter\def\csname PY@tok@vm\endcsname{\def\PY@tc##1{\textcolor[rgb]{0.10,0.09,0.49}{##1}}}
\expandafter\def\csname PY@tok@sa\endcsname{\def\PY@tc##1{\textcolor[rgb]{0.73,0.13,0.13}{##1}}}
\expandafter\def\csname PY@tok@sb\endcsname{\def\PY@tc##1{\textcolor[rgb]{0.73,0.13,0.13}{##1}}}
\expandafter\def\csname PY@tok@sc\endcsname{\def\PY@tc##1{\textcolor[rgb]{0.73,0.13,0.13}{##1}}}
\expandafter\def\csname PY@tok@dl\endcsname{\def\PY@tc##1{\textcolor[rgb]{0.73,0.13,0.13}{##1}}}
\expandafter\def\csname PY@tok@s2\endcsname{\def\PY@tc##1{\textcolor[rgb]{0.73,0.13,0.13}{##1}}}
\expandafter\def\csname PY@tok@sh\endcsname{\def\PY@tc##1{\textcolor[rgb]{0.73,0.13,0.13}{##1}}}
\expandafter\def\csname PY@tok@s1\endcsname{\def\PY@tc##1{\textcolor[rgb]{0.73,0.13,0.13}{##1}}}
\expandafter\def\csname PY@tok@mb\endcsname{\def\PY@tc##1{\textcolor[rgb]{0.40,0.40,0.40}{##1}}}
\expandafter\def\csname PY@tok@mf\endcsname{\def\PY@tc##1{\textcolor[rgb]{0.40,0.40,0.40}{##1}}}
\expandafter\def\csname PY@tok@mh\endcsname{\def\PY@tc##1{\textcolor[rgb]{0.40,0.40,0.40}{##1}}}
\expandafter\def\csname PY@tok@mi\endcsname{\def\PY@tc##1{\textcolor[rgb]{0.40,0.40,0.40}{##1}}}
\expandafter\def\csname PY@tok@il\endcsname{\def\PY@tc##1{\textcolor[rgb]{0.40,0.40,0.40}{##1}}}
\expandafter\def\csname PY@tok@mo\endcsname{\def\PY@tc##1{\textcolor[rgb]{0.40,0.40,0.40}{##1}}}
\expandafter\def\csname PY@tok@ch\endcsname{\let\PY@it=\textit\def\PY@tc##1{\textcolor[rgb]{0.25,0.50,0.50}{##1}}}
\expandafter\def\csname PY@tok@cm\endcsname{\let\PY@it=\textit\def\PY@tc##1{\textcolor[rgb]{0.25,0.50,0.50}{##1}}}
\expandafter\def\csname PY@tok@cpf\endcsname{\let\PY@it=\textit\def\PY@tc##1{\textcolor[rgb]{0.25,0.50,0.50}{##1}}}
\expandafter\def\csname PY@tok@c1\endcsname{\let\PY@it=\textit\def\PY@tc##1{\textcolor[rgb]{0.25,0.50,0.50}{##1}}}
\expandafter\def\csname PY@tok@cs\endcsname{\let\PY@it=\textit\def\PY@tc##1{\textcolor[rgb]{0.25,0.50,0.50}{##1}}}

\def\PYZbs{\char`\\}
\def\PYZus{\char`\_}
\def\PYZob{\char`\{}
\def\PYZcb{\char`\}}
\def\PYZca{\char`\^}
\def\PYZam{\char`\&}
\def\PYZlt{\char`\<}
\def\PYZgt{\char`\>}
\def\PYZsh{\char`\#}
\def\PYZpc{\char`\%}
\def\PYZdl{\char`\$}
\def\PYZhy{\char`\-}
\def\PYZsq{\char`\'}
\def\PYZdq{\char`\"}
\def\PYZti{\char`\~}
% for compatibility with earlier versions
\def\PYZat{@}
\def\PYZlb{[}
\def\PYZrb{]}
\makeatother


    % Exact colors from NB
    \definecolor{incolor}{rgb}{0.0, 0.0, 0.5}
    \definecolor{outcolor}{rgb}{0.545, 0.0, 0.0}



    
    % Prevent overflowing lines due to hard-to-break entities
    \sloppy 
    % Setup hyperref package
    \hypersetup{
      breaklinks=true,  % so long urls are correctly broken across lines
      colorlinks=true,
      urlcolor=urlcolor,
      linkcolor=linkcolor,
      citecolor=citecolor,
      }
    % Slightly bigger margins than the latex defaults
    
    \geometry{verbose,tmargin=1in,bmargin=1in,lmargin=1in,rmargin=1in}
    
    

    \begin{document}
    
    
    \maketitle
    
    

    
    Hazırlayan

Versiyon

Tarih

Dil

dataFLOYD

v1.00

20 Şub 2019

TR

    \hypertarget{temel-r-kullanux131mux131}{%
\section{Temel R kullanımı}\label{temel-r-kullanux131mux131}}

    \begin{figure}
\centering
\includegraphics{r_basicslogo.png}
\caption{r\_basicslogo.png}
\end{figure}

    Evet, \texttt{dataFLOYD} olarak R programlama dilini de ihmal etmek
istemiyoruz. Bu yazımızla bir başlangıç verelim gerisi gelir herhalde :)
Lafı uzatmadan hemen konumuza başlayalım

    \hypertarget{bilgisayarux131mux131za-nasux131l-r-kurarux131z}{%
\subsection{Bilgisayarımıza nasıl R
kurarız?}\label{bilgisayarux131mux131za-nasux131l-r-kurarux131z}}

    Başlamadan önce bilgisayarınıza ilk kez R kuracaksanız aşağıdaki
yöntemleri kullanabilirsiniz.

    \hypertarget{yuxf6ntem-rstudio-ile-r-kullanux131mux131}{%
\subsubsection{\texorpdfstring{1. Yöntem: RStudio ile \texttt{R}
kullanımı}{1. Yöntem: RStudio ile R kullanımı}}\label{yuxf6ntem-rstudio-ile-r-kullanux131mux131}}

    Öncelikle aşağıdaki bağlantıdan işletim sisteminize uygun olan
\texttt{R} paketini indirin ve bilgisayarınıza kurulumu gerçekleştirin.

    https://cran.r-project.org/

    Rstudio Desktop programını indirin. ``Open Source License'' versiyonunu
kullanın. Bilgisayarınıza kurulum yapın

    https://www.rstudio.com/products/rstudio/download/

    RStudio programını kullanarak \texttt{R} alemlerine akabilirsiniz artık
tebrikler!

    \hypertarget{yuxf6ntem-anaconda-uxfczerinden-kurulum-yaparak-jupyter-notebook-ile-kullanmak}{%
\subsubsection{2. Yöntem: Anaconda üzerinden kurulum yaparak Jupyter
Notebook ile
kullanmak}\label{yuxf6ntem-anaconda-uxfczerinden-kurulum-yaparak-jupyter-notebook-ile-kullanmak}}

    Bilgisayarınızda Anaconda yoksa ilk önce bunu kurmalısınız. Python 3.x
versiyonunu tercih etmenizi tavsiye ederim.

    https://www.anaconda.com/distribution/\#download-section

    Anaconda kurulumu sonrası ``Anaconda Prompt'' uygulamasını açınız ve
aşağıdaki komutu yazarak \texttt{R} için yeni bir ortam oluşturun

    \texttt{conda\ create\ -n\ mro\_env\ r-essentials\ mro-base}

    Komut sonrası kurulum biraz vakit alabilir sabırlı olun:)

    Herşey tamamlanınca yine ``Anaconda Prompt'' uygulamasında
oluşturduğumuz ortama geçiş yapın.

    \texttt{conda\ activate\ mro\_env}

    Sonrasında yine ``Anaconda Prompt''ta aşağıdaki komutu giriyoruz ve
``browser''da ``Jupyter'' açılıyor

    \texttt{jupyter\ notebook}

    Burada ``New'' kısmına basın, \texttt{R} seçin ve artık hazırsınız!

    \begin{center}\rule{0.5\linewidth}{\linethickness}\end{center}

    \hypertarget{temel-operasyonlar}{%
\subsection{Temel operasyonlar}\label{temel-operasyonlar}}

    Temel operasyonlar her dilde hemen hemen aynıdır. 4 işleme bakalım

    \begin{quote}
R dilinde ``comment'' ekleme \# ile yapılır
\end{quote}

    \begin{Verbatim}[commandchars=\\\{\}]
{\color{incolor}In [{\color{incolor}1}]:} \PY{c+c1}{\PYZsh{} Toplama}
        \PY{l+m}{42} \PY{o}{+} \PY{l+m}{42}
        \PY{c+c1}{\PYZsh{} Çıkarma}
        \PY{l+m}{42}\PY{l+m}{\PYZhy{}12}
        \PY{c+c1}{\PYZsh{} Çarpma}
        \PY{l+m}{2} \PY{o}{*} \PY{l+m}{2}
        \PY{c+c1}{\PYZsh{} Bölme}
        \PY{l+m}{12} \PY{o}{/} \PY{l+m}{5}
        \PY{c+c1}{\PYZsh{} Üs alma}
        \PY{l+m}{5}\PY{n}{\PYZca{}3}
        \PY{c+c1}{\PYZsh{} Mod alma}
        \PY{l+m}{13} \PY{o}{\PYZpc{}\PYZpc{}} \PY{l+m}{3}
\end{Verbatim}


    84

    
    30

    
    4

    
    2.4

    
    125

    
    1

    
    \hypertarget{deux11fiux15fken-tanux131mlama-kurallarux131}{%
\subsection{Değişken tanımlama
kuralları}\label{deux11fiux15fken-tanux131mlama-kurallarux131}}

    Değişken tanımlanırken aşağıdaki kurallara dikkat edilir * Değişkenlerde
harf, rakam ve \texttt{.}, \texttt{\_} bulunabilir. * Örnek:
\texttt{elma}, \texttt{armut20}, \texttt{benim.degiskenim},
\texttt{bu\_da\_olur}, \texttt{Büyük.Harf} * Değişkenler bir harf veya
\texttt{.} le başlamalı. Eğer \texttt{.} ile başlarsa bunun ardından bir
rakam gelemez * Örnek: \texttt{.degisken}, \texttt{r2d2} * Bazı
\texttt{reserved} kelimeler değişken ismi olarak kullanılamaz * Örnek:
\texttt{TRUE}, \texttt{NULL}, \texttt{NaN} değişken olarak kullanılamaz

    Şimdi bu R dilinde bir değişkene değer atama (assignment) dediğimiz olay
biraz değişiktir. Python gibi bir dil kullanıyorsanız genellikle
değişkene değer atamak için \texttt{=} kullanılır biliyorsunuz.

    Ama \texttt{R} dilinde genel kabul edilen kullanım \texttt{\textless{}-}
şeklindedir. \texttt{=} operatörü de çalışır, fakat \texttt{R}
kullanıyorsanız \texttt{\textless{}-} kullanın lütfen, daha havalı olur
:)

    \begin{Verbatim}[commandchars=\\\{\}]
{\color{incolor}In [{\color{incolor}2}]:} \PY{n}{a} \PY{o}{\PYZlt{}\PYZhy{}} \PY{l+m}{3}\PY{o}{*}\PY{l+m}{3}
        \PY{n}{a}
\end{Verbatim}


    9

    
    \begin{Verbatim}[commandchars=\\\{\}]
{\color{incolor}In [{\color{incolor}3}]:} \PY{c+c1}{\PYZsh{} = operatörü de çalışır ama bunu kullanmayı tavsiye etmem}
        \PY{n}{a} \PY{o}{=} \PY{l+m}{3}\PY{o}{*}\PY{l+m}{3}
        \PY{n}{a}
\end{Verbatim}


    9

    
    \begin{Verbatim}[commandchars=\\\{\}]
{\color{incolor}In [{\color{incolor}4}]:} \PY{n}{a} \PY{o}{\PYZlt{}\PYZhy{}} \PY{l+m}{3}
        \PY{n}{b} \PY{o}{\PYZlt{}\PYZhy{}} \PY{l+m}{2}
        \PY{n}{c} \PY{o}{\PYZlt{}\PYZhy{}} \PY{n}{a} \PY{o}{+}\PY{n}{b}
        \PY{n}{c}
\end{Verbatim}


    5

    
    \hypertarget{string}{%
\subsubsection{string}\label{string}}

    \begin{Verbatim}[commandchars=\\\{\}]
{\color{incolor}In [{\color{incolor}5}]:} \PY{n}{ilk\PYZus{}str} \PY{o}{\PYZlt{}\PYZhy{}} \PY{l+s}{\PYZdq{}}\PY{l+s}{Merhaba dünya!\PYZdq{}}
        \PY{n}{ilk\PYZus{}str}
\end{Verbatim}


    'Merhaba dünya!'

    
    \hypertarget{logical}{%
\subsubsection{logical}\label{logical}}

    \begin{Verbatim}[commandchars=\\\{\}]
{\color{incolor}In [{\color{incolor}6}]:} \PY{n}{asd} \PY{o}{\PYZlt{}\PYZhy{}} \PY{k+kc}{TRUE}
        \PY{n}{qwe} \PY{o}{\PYZlt{}\PYZhy{}} \PY{k+kc}{FALSE}
        \PY{n}{asd}
        \PY{n}{qwe}
\end{Verbatim}


    TRUE

    
    FALSE

    
    \hypertarget{deux11fiux15fkenlerin-tiplerini-uxf6ux11frenmek}{%
\subsubsection{Değişkenlerin tiplerini
öğrenmek}\label{deux11fiux15fkenlerin-tiplerini-uxf6ux11frenmek}}

    Bu amaçla \texttt{class} fonksiyonu kullanılır.

    \begin{Verbatim}[commandchars=\\\{\}]
{\color{incolor}In [{\color{incolor}7}]:} \PY{n}{x} \PY{o}{\PYZlt{}\PYZhy{}} \PY{l+m}{3.14}
        \PY{n+nf}{class}\PY{p}{(}\PY{n}{x}\PY{p}{)}
        \PY{n}{y} \PY{o}{\PYZlt{}\PYZhy{}} \PY{l+s}{\PYZdq{}}\PY{l+s}{pi\PYZdq{}}
        \PY{n+nf}{class}\PY{p}{(}\PY{n}{y}\PY{p}{)}
        \PY{n}{z} \PY{o}{\PYZlt{}\PYZhy{}} \PY{k+kc}{FALSE}
        \PY{n+nf}{class}\PY{p}{(}\PY{n}{z}\PY{p}{)}
\end{Verbatim}


    'numeric'

    
    'character'

    
    'logical'

    
    \hypertarget{vektuxf6rler}{%
\subsection{Vektörler}\label{vektuxf6rler}}

    \texttt{R} kullanacaksanız vektörleri mutlaka bilmeniz gerekir.
\texttt{c()} fonksiyonu kullanılarak vektör oluşturulur. Vektörler sayı
olabileceği gibi başka tiplerden de oluşabilirler

    \begin{Verbatim}[commandchars=\\\{\}]
{\color{incolor}In [{\color{incolor}8}]:} \PY{c+c1}{\PYZsh{} Nümerik}
        \PY{n}{a} \PY{o}{\PYZlt{}\PYZhy{}} \PY{n+nf}{c}\PY{p}{(}\PY{l+m}{1}\PY{p}{,}\PY{l+m}{2}\PY{p}{,}\PY{l+m}{3}\PY{p}{)}
        \PY{n}{a}
        \PY{c+c1}{\PYZsh{} string}
        \PY{n}{b} \PY{o}{\PYZlt{}\PYZhy{}} \PY{n+nf}{c}\PY{p}{(}\PY{l+s}{\PYZdq{}}\PY{l+s}{bir\PYZdq{}}\PY{p}{,}\PY{l+s}{\PYZdq{}}\PY{l+s}{iki\PYZdq{}}\PY{p}{,}\PY{l+s}{\PYZdq{}}\PY{l+s}{üç\PYZdq{}}\PY{p}{)}
        \PY{n}{b}
\end{Verbatim}


    \begin{enumerate*}
\item 1
\item 2
\item 3
\end{enumerate*}


    
    \begin{enumerate*}
\item 'bir'
\item 'iki'
\item 'üç'
\end{enumerate*}


    
    \hypertarget{vektuxf6rlerin-her-bir-elemanux131na-isim-verilebilir}{%
\subsubsection{Vektörlerin her bir elemanına isim
verilebilir}\label{vektuxf6rlerin-her-bir-elemanux131na-isim-verilebilir}}

    \texttt{names()} fonksiyonu kullanılarak bu işlem gerçekleştirilebilir.

    \begin{Verbatim}[commandchars=\\\{\}]
{\color{incolor}In [{\color{incolor}9}]:} \PY{n}{a} \PY{o}{\PYZlt{}\PYZhy{}} \PY{n+nf}{c}\PY{p}{(}\PY{l+m}{10}\PY{p}{,}\PY{l+m}{20}\PY{p}{,}\PY{l+m}{30}\PY{p}{)}
        \PY{n}{isimler} \PY{o}{\PYZlt{}\PYZhy{}} \PY{n+nf}{c}\PY{p}{(}\PY{l+s}{\PYZdq{}}\PY{l+s}{Elma\PYZdq{}}\PY{p}{,}\PY{l+s}{\PYZdq{}}\PY{l+s}{Armut\PYZdq{}}\PY{p}{,}\PY{l+s}{\PYZdq{}}\PY{l+s}{Portakal\PYZdq{}}\PY{p}{)}
        \PY{n+nf}{names}\PY{p}{(}\PY{n}{a}\PY{p}{)} \PY{o}{\PYZlt{}\PYZhy{}} \PY{n}{isimler}
        \PY{n}{a}
        \PY{c+c1}{\PYZsh{} İsimlemdirilmiş alanlara erişim}
        \PY{n}{a}\PY{n}{[}\PY{l+s}{\PYZdq{}}\PY{l+s}{Elma\PYZdq{}}\PY{n}{]}
\end{Verbatim}


    \begin{description*}
\item[Elma] 10
\item[Armut] 20
\item[Portakal] 30
\end{description*}


    
    \textbf{Elma:} 10

    
    \hypertarget{nuxfcmerik-vektuxf6rler-uxfczerinde-uxe7eux15fitli-iux15flemler}{%
\subsubsection{Nümerik vektörler üzerinde çeşitli
işlemler}\label{nuxfcmerik-vektuxf6rler-uxfczerinde-uxe7eux15fitli-iux15flemler}}

İşlemler eleman bazında yapılmaktadır.

    \begin{quote}
\textbf{İki vektörü eleman bazında toplamak, çarpmak, çıkarmak vb..}
\end{quote}

    \begin{Verbatim}[commandchars=\\\{\}]
{\color{incolor}In [{\color{incolor}10}]:} \PY{n}{a} \PY{o}{\PYZlt{}\PYZhy{}} \PY{n+nf}{c}\PY{p}{(}\PY{l+m}{10}\PY{p}{,}\PY{l+m}{20}\PY{p}{,}\PY{l+m}{30}\PY{p}{)}
         \PY{n}{b} \PY{o}{\PYZlt{}\PYZhy{}} \PY{n+nf}{c}\PY{p}{(}\PY{l+m}{1}\PY{p}{,}\PY{l+m}{2}\PY{p}{,}\PY{l+m}{3}\PY{p}{)}
         \PY{n}{a}\PY{o}{+}\PY{n}{b}
         \PY{n}{a}\PY{o}{*}\PY{n}{b}
         \PY{n}{a}\PY{n}{\PYZca{}b}
\end{Verbatim}


    \begin{enumerate*}
\item 11
\item 22
\item 33
\end{enumerate*}


    
    \begin{enumerate*}
\item 10
\item 40
\item 90
\end{enumerate*}


    
    \begin{enumerate*}
\item 10
\item 400
\item 27000
\end{enumerate*}


    
    \begin{Verbatim}[commandchars=\\\{\}]
{\color{incolor}In [{\color{incolor}11}]:} \PY{c+c1}{\PYZsh{} Boyutlar tutmazsa hata (error) vermez}
         \PY{c+c1}{\PYZsh{} kısa olan vektörü ilk elemanından başlayarak }
         \PY{c+c1}{\PYZsh{} diğerinin boyutuna yetişecek şekilde tamamlar}
         \PY{n}{a} \PY{o}{\PYZlt{}\PYZhy{}} \PY{n+nf}{c}\PY{p}{(}\PY{l+m}{10}\PY{p}{,}\PY{l+m}{20}\PY{p}{,}\PY{l+m}{30}\PY{p}{)}
         \PY{n}{b} \PY{o}{\PYZlt{}\PYZhy{}} \PY{n+nf}{c}\PY{p}{(}\PY{l+m}{1}\PY{p}{,}\PY{l+m}{2}\PY{p}{,}\PY{l+m}{3}\PY{p}{,}\PY{l+m}{4}\PY{p}{)}
         \PY{n}{a}\PY{o}{+}\PY{n}{b}
\end{Verbatim}


    \begin{Verbatim}[commandchars=\\\{\}]
Warning message in a + b:
"uzun olan nesne uzunluğu kısa olan nesne uzunluğunun bir katı değil "
    \end{Verbatim}

    \begin{enumerate*}
\item 11
\item 22
\item 33
\item 14
\end{enumerate*}


    
    \begin{quote}
\textbf{Bir vektörün elemanlarının toplamı ve ortalaması}
\end{quote}

    Toplama için \texttt{sum()} ve ortalama için \texttt{mean()}
fonksiyonları kullanılır.

    \begin{Verbatim}[commandchars=\\\{\}]
{\color{incolor}In [{\color{incolor}12}]:} \PY{n}{a} \PY{o}{\PYZlt{}\PYZhy{}} \PY{n+nf}{c}\PY{p}{(}\PY{l+m}{1}\PY{p}{,}\PY{l+m}{1}\PY{p}{,}\PY{l+m}{2}\PY{p}{,}\PY{l+m}{3}\PY{p}{,}\PY{l+m}{5}\PY{p}{,}\PY{l+m}{8}\PY{p}{)}
         \PY{n+nf}{sum}\PY{p}{(}\PY{n}{a}\PY{p}{)}
         \PY{n+nf}{mean}\PY{p}{(}\PY{n}{a}\PY{p}{)}
\end{Verbatim}


    20

    
    3.33333333333333

    
    \begin{quote}
\textbf{İki vektörü karşılaştırma}
\end{quote}

    \begin{Verbatim}[commandchars=\\\{\}]
{\color{incolor}In [{\color{incolor}13}]:} \PY{n}{a} \PY{o}{\PYZlt{}\PYZhy{}} \PY{n+nf}{c}\PY{p}{(}\PY{l+m}{1}\PY{p}{,}\PY{l+m}{1}\PY{p}{,}\PY{l+m}{2}\PY{p}{,}\PY{l+m}{3}\PY{p}{,}\PY{l+m}{5}\PY{p}{,}\PY{l+m}{8}\PY{p}{)}
         \PY{n}{b} \PY{o}{\PYZlt{}\PYZhy{}} \PY{n+nf}{c}\PY{p}{(}\PY{l+m}{3}\PY{p}{,}\PY{l+m}{3}\PY{p}{,}\PY{l+m}{3}\PY{p}{,}\PY{l+m}{0}\PY{p}{,}\PY{l+m}{0}\PY{p}{,}\PY{l+m}{0}\PY{p}{)}
         \PY{n}{a}\PY{o}{\PYZlt{}}\PY{n}{b}
         \PY{c+c1}{\PYZsh{} a\PYZsq{}nın b\PYZsq{}den küçük elemanları için TRUE, }
         \PY{c+c1}{\PYZsh{} diğerleri için FALSE olan bir vektör döndürür}
\end{Verbatim}


    \begin{enumerate*}
\item TRUE
\item TRUE
\item TRUE
\item FALSE
\item FALSE
\item FALSE
\end{enumerate*}


    
    \begin{quote}
\textbf{Vektörün elemanlarına erişim}
\end{quote}

    \texttt{R} programlama dilinde indeksler 1'den başlar. İstediğimiz
elemana erişmek için \texttt{{[}{]}} kullanılır

    \begin{Verbatim}[commandchars=\\\{\}]
{\color{incolor}In [{\color{incolor}14}]:} \PY{n}{a} \PY{o}{\PYZlt{}\PYZhy{}} \PY{n+nf}{c}\PY{p}{(}\PY{l+m}{1}\PY{p}{,}\PY{l+m}{2}\PY{p}{,}\PY{l+m}{3}\PY{p}{,}\PY{l+m}{4}\PY{p}{,}\PY{l+m}{5}\PY{p}{)}
         \PY{n}{a}\PY{n}{[1}\PY{n}{]}
         \PY{n}{a}\PY{n}{[5}\PY{n}{]}
\end{Verbatim}


    1

    
    5

    
    İndekslemeyi vektör olarak da yapabiliriz, bunun için bir vektör
oluşturmamız gerekiyor. Yani bir serinin 1. ,3. ve 5. elemanına erişmek
istersek bir vektör kullanmamız lazım

    \begin{Verbatim}[commandchars=\\\{\}]
{\color{incolor}In [{\color{incolor}15}]:} \PY{n}{a} \PY{o}{\PYZlt{}\PYZhy{}} \PY{n+nf}{c}\PY{p}{(}\PY{l+m}{1}\PY{p}{,}\PY{l+m}{2}\PY{p}{,}\PY{l+m}{3}\PY{p}{,}\PY{l+m}{4}\PY{p}{,}\PY{l+m}{5}\PY{p}{)}
         \PY{c+c1}{\PYZsh{} a vektörünün 1,3,5 elemanlarına erişmek için}
         \PY{n}{a}\PY{n+nf}{[c}\PY{p}{(}\PY{l+m}{1}\PY{p}{,}\PY{l+m}{3}\PY{p}{,}\PY{l+m}{5}\PY{p}{)}\PY{n}{]}
\end{Verbatim}


    \begin{enumerate*}
\item 1
\item 3
\item 5
\end{enumerate*}


    
    \begin{quote}
\textbf{Kesit (Slice) alma}
\end{quote}

    Mesela \texttt{a} değişkeninin 2. elemanından 4. elemanına erişmek
istersek \texttt{a{[}2:4{]}} kullanılır

    \begin{Verbatim}[commandchars=\\\{\}]
{\color{incolor}In [{\color{incolor}16}]:} \PY{n}{a} \PY{o}{\PYZlt{}\PYZhy{}} \PY{n+nf}{c}\PY{p}{(}\PY{l+m}{1}\PY{p}{,}\PY{l+m}{2}\PY{p}{,}\PY{l+m}{3}\PY{p}{,}\PY{l+m}{4}\PY{p}{,}\PY{l+m}{5}\PY{p}{)}
         \PY{c+c1}{\PYZsh{} a vektörünün 1,3,5 elemanlarına erişmek için}
         \PY{n}{a}\PY{n}{[2}\PY{o}{:}\PY{l+m}{4}\PY{n}{]}
\end{Verbatim}


    \begin{enumerate*}
\item 2
\item 3
\item 4
\end{enumerate*}


    
    \hypertarget{matrisler}{%
\subsection{Matrisler}\label{matrisler}}

    Matrisler iki boyutlu vektörlerdir bildiğiniz gibi. Matris yaratmak için
\texttt{matrix()} fonksiyonunu kullanırız.

    \begin{Verbatim}[commandchars=\\\{\}]
{\color{incolor}In [{\color{incolor}17}]:} \PY{c+c1}{\PYZsh{} Bu fonksiyonu çağırdığınızda}
         \PY{c+c1}{\PYZsh{} 1:9 1\PYZsq{}den 9\PYZsq{}a kadar bir vektör yaratır}
         \PY{c+c1}{\PYZsh{} nrow: Bu vektörü 3 satıra böler }
         \PY{n+nf}{matrix}\PY{p}{(}\PY{l+m}{1}\PY{o}{:}\PY{l+m}{9}\PY{p}{,}\PY{n}{nrow}\PY{o}{=}\PY{l+m}{3}\PY{p}{)}
\end{Verbatim}


    \begin{tabular}{lll}
	 1 & 4 & 7\\
	 2 & 5 & 8\\
	 3 & 6 & 9\\
\end{tabular}


    
    \begin{Verbatim}[commandchars=\\\{\}]
{\color{incolor}In [{\color{incolor}18}]:} \PY{c+c1}{\PYZsh{} Vektörün sıralamasını satır bazında yapmak istersek}
         \PY{c+c1}{\PYZsh{} byrow=TRUE yapmamız gerekir}
         \PY{n+nf}{matrix}\PY{p}{(}\PY{l+m}{1}\PY{o}{:}\PY{l+m}{9}\PY{p}{,}\PY{n}{nrow}\PY{o}{=}\PY{l+m}{3}\PY{p}{,}\PY{n}{byrow}\PY{o}{=}\PY{k+kc}{TRUE}\PY{p}{)}
\end{Verbatim}


    \begin{tabular}{lll}
	 1 & 2 & 3\\
	 4 & 5 & 6\\
	 7 & 8 & 9\\
\end{tabular}


    
    \begin{Verbatim}[commandchars=\\\{\}]
{\color{incolor}In [{\color{incolor}19}]:} \PY{c+c1}{\PYZsh{} Matrislerin satır ve sütunlarına isim verilebilir}
         \PY{n}{a} \PY{o}{\PYZlt{}\PYZhy{}} \PY{n+nf}{matrix}\PY{p}{(}\PY{l+m}{1}\PY{o}{:}\PY{l+m}{9}\PY{p}{,}\PY{n}{nrow}\PY{o}{=}\PY{l+m}{3}\PY{p}{,}\PY{n}{byrow}\PY{o}{=}\PY{k+kc}{TRUE}\PY{p}{)}
         \PY{n+nf}{colnames}\PY{p}{(}\PY{n}{a}\PY{p}{)} \PY{o}{\PYZlt{}\PYZhy{}} \PY{n+nf}{c}\PY{p}{(}\PY{l+s}{\PYZdq{}}\PY{l+s}{col1\PYZdq{}}\PY{p}{,}\PY{l+s}{\PYZdq{}}\PY{l+s}{col2\PYZdq{}}\PY{p}{,}\PY{l+s}{\PYZdq{}}\PY{l+s}{col3\PYZdq{}}\PY{p}{)}
         \PY{n+nf}{rownames}\PY{p}{(}\PY{n}{a}\PY{p}{)} \PY{o}{\PYZlt{}\PYZhy{}} \PY{n+nf}{c}\PY{p}{(}\PY{l+s}{\PYZdq{}}\PY{l+s}{row1\PYZdq{}}\PY{p}{,}\PY{l+s}{\PYZdq{}}\PY{l+s}{row2\PYZdq{}}\PY{p}{,}\PY{l+s}{\PYZdq{}}\PY{l+s}{row3\PYZdq{}}\PY{p}{)}
         \PY{n}{a}
\end{Verbatim}


    \begin{tabular}{r|lll}
  & col1 & col2 & col3\\
\hline
	row1 & 1 & 2 & 3\\
	row2 & 4 & 5 & 6\\
	row3 & 7 & 8 & 9\\
\end{tabular}


    
    \begin{Verbatim}[commandchars=\\\{\}]
{\color{incolor}In [{\color{incolor}20}]:} \PY{c+c1}{\PYZsh{} isimlendirme yaparsak erişim için bunu da kullanabiliriz}
         \PY{n}{a}\PY{n}{[}\PY{l+s}{\PYZdq{}}\PY{l+s}{row1\PYZdq{}}\PY{p}{,}\PY{l+s}{\PYZdq{}}\PY{l+s}{col2\PYZdq{}}\PY{n}{]}
\end{Verbatim}


    2

    
    \begin{quote}
\textbf{Satır ve sütun toplamlarının alınması}
\end{quote}

    \begin{Verbatim}[commandchars=\\\{\}]
{\color{incolor}In [{\color{incolor}21}]:} \PY{n}{a} \PY{o}{\PYZlt{}\PYZhy{}} \PY{n+nf}{matrix}\PY{p}{(}\PY{l+m}{1}\PY{o}{:}\PY{l+m}{9}\PY{p}{,}\PY{n}{nrow}\PY{o}{=}\PY{l+m}{3}\PY{p}{,}\PY{n}{byrow}\PY{o}{=}\PY{k+kc}{TRUE}\PY{p}{)}
         \PY{c+c1}{\PYZsh{} satır bazında toplam}
         \PY{n+nf}{rowSums}\PY{p}{(}\PY{n}{a}\PY{p}{)}
         \PY{c+c1}{\PYZsh{} Burada kafanız karışmasın sadece vektörü yatay olarak gösterdi }
         
         \PY{c+c1}{\PYZsh{} sütun bazında toplam}
         \PY{n+nf}{colSums}\PY{p}{(}\PY{n}{a}\PY{p}{)}
\end{Verbatim}


    \begin{enumerate*}
\item 6
\item 15
\item 24
\end{enumerate*}


    
    \begin{enumerate*}
\item 12
\item 15
\item 18
\end{enumerate*}


    
    \begin{quote}
Bir matrise vektör/matris ekleme \texttt{cbind()} ve \texttt{rbind()}
\end{quote}

    \begin{Verbatim}[commandchars=\\\{\}]
{\color{incolor}In [{\color{incolor}22}]:} \PY{n}{a} \PY{o}{\PYZlt{}\PYZhy{}} \PY{n+nf}{matrix}\PY{p}{(}\PY{l+m}{1}\PY{o}{:}\PY{l+m}{9}\PY{p}{,}\PY{n}{nrow}\PY{o}{=}\PY{l+m}{3}\PY{p}{,}\PY{n}{byrow}\PY{o}{=}\PY{k+kc}{TRUE}\PY{p}{)}
         \PY{n}{a}
\end{Verbatim}


    \begin{tabular}{lll}
	 1 & 2 & 3\\
	 4 & 5 & 6\\
	 7 & 8 & 9\\
\end{tabular}


    
    \begin{Verbatim}[commandchars=\\\{\}]
{\color{incolor}In [{\color{incolor}23}]:} \PY{n+nf}{cbind}\PY{p}{(}\PY{n}{a}\PY{p}{,}\PY{n+nf}{c}\PY{p}{(}\PY{l+m}{10}\PY{p}{,}\PY{l+m}{11}\PY{p}{,}\PY{l+m}{12}\PY{p}{)}\PY{p}{)}
\end{Verbatim}


    \begin{tabular}{llll}
	 1  & 2  & 3  & 10\\
	 4  & 5  & 6  & 11\\
	 7  & 8  & 9  & 12\\
\end{tabular}


    
    \begin{Verbatim}[commandchars=\\\{\}]
{\color{incolor}In [{\color{incolor}24}]:} \PY{n+nf}{rbind}\PY{p}{(}\PY{n}{a}\PY{p}{,}\PY{n+nf}{c}\PY{p}{(}\PY{l+m}{10}\PY{p}{,}\PY{l+m}{11}\PY{p}{,}\PY{l+m}{12}\PY{p}{)}\PY{p}{)}
\end{Verbatim}


    \begin{tabular}{lll}
	  1 &  2 &  3\\
	  4 &  5 &  6\\
	  7 &  8 &  9\\
	 10 & 11 & 12\\
\end{tabular}


    
    \begin{Verbatim}[commandchars=\\\{\}]
{\color{incolor}In [{\color{incolor}25}]:} \PY{c+c1}{\PYZsh{} iki matrisi birbirine ekleme}
         \PY{n}{b} \PY{o}{\PYZlt{}\PYZhy{}} \PY{n+nf}{matrix}\PY{p}{(}\PY{l+m}{11}\PY{o}{:}\PY{l+m}{19}\PY{p}{,}\PY{n}{nrow}\PY{o}{=}\PY{l+m}{3}\PY{p}{,}\PY{n}{byrow}\PY{o}{=}\PY{k+kc}{TRUE}\PY{p}{)}
         \PY{n+nf}{cbind}\PY{p}{(}\PY{n}{a}\PY{p}{,}\PY{n}{b}\PY{p}{)}
\end{Verbatim}


    \begin{tabular}{llllll}
	 1  & 2  & 3  & 11 & 12 & 13\\
	 4  & 5  & 6  & 14 & 15 & 16\\
	 7  & 8  & 9  & 17 & 18 & 19\\
\end{tabular}


    
    \begin{quote}
\textbf{Matrislerden kesit alma}
\end{quote}

    \begin{Verbatim}[commandchars=\\\{\}]
{\color{incolor}In [{\color{incolor}26}]:} \PY{n}{a} \PY{o}{\PYZlt{}\PYZhy{}} \PY{n+nf}{matrix}\PY{p}{(}\PY{l+m}{1}\PY{o}{:}\PY{l+m}{9}\PY{p}{,}\PY{n}{nrow}\PY{o}{=}\PY{l+m}{3}\PY{p}{,}\PY{n}{byrow}\PY{o}{=}\PY{k+kc}{TRUE}\PY{p}{)}
         \PY{n}{a}
         \PY{c+c1}{\PYZsh{} a matrisinden kesit}
         \PY{n}{a}\PY{n}{[2}\PY{o}{:}\PY{l+m}{3}\PY{p}{,}\PY{l+m}{2}\PY{o}{:}\PY{l+m}{3}\PY{n}{]}
\end{Verbatim}


    \begin{tabular}{lll}
	 1 & 2 & 3\\
	 4 & 5 & 6\\
	 7 & 8 & 9\\
\end{tabular}


    
    \begin{tabular}{ll}
	 5 & 6\\
	 8 & 9\\
\end{tabular}


    
    \begin{Verbatim}[commandchars=\\\{\}]
{\color{incolor}In [{\color{incolor}27}]:} \PY{c+c1}{\PYZsh{} a matrisinin ilk sütunu}
         \PY{n}{a}\PY{n}{[}\PY{p}{,}\PY{l+m}{1}\PY{n}{]}
\end{Verbatim}


    \begin{enumerate*}
\item 1
\item 4
\item 7
\end{enumerate*}


    
    \begin{Verbatim}[commandchars=\\\{\}]
{\color{incolor}In [{\color{incolor}28}]:} \PY{c+c1}{\PYZsh{} a matrisinin ilk satırı}
         \PY{n}{a}\PY{n}{[1}\PY{p}{,}\PY{n}{]}
\end{Verbatim}


    \begin{enumerate*}
\item 1
\item 2
\item 3
\end{enumerate*}


    
    \hypertarget{faktuxf6r-factors}{%
\subsection{Faktör (Factors)}\label{faktuxf6r-factors}}

    Dikkatinizi istiyorum burada, eğer kategorik bir değişken yaratmak
istiyorsanız (örneğin Kadın Erkek gibi) \texttt{factors()} fonksiyonunu
kullanılırız.

    \begin{Verbatim}[commandchars=\\\{\}]
{\color{incolor}In [{\color{incolor}29}]:} \PY{n}{a} \PY{o}{\PYZlt{}\PYZhy{}} \PY{n+nf}{c}\PY{p}{(}\PY{l+s}{\PYZdq{}}\PY{l+s}{K\PYZdq{}}\PY{p}{,}\PY{l+s}{\PYZdq{}}\PY{l+s}{E\PYZdq{}}\PY{p}{,}\PY{l+s}{\PYZdq{}}\PY{l+s}{E\PYZdq{}}\PY{p}{,}\PY{l+s}{\PYZdq{}}\PY{l+s}{K\PYZdq{}}\PY{p}{,}\PY{l+s}{\PYZdq{}}\PY{l+s}{K\PYZdq{}}\PY{p}{)}
         \PY{n+nf}{class}\PY{p}{(}\PY{n}{a}\PY{p}{)}
         \PY{n}{b} \PY{o}{\PYZlt{}\PYZhy{}} \PY{n+nf}{factor}\PY{p}{(}\PY{n}{a}\PY{p}{)}
         \PY{n+nf}{class}\PY{p}{(}\PY{n}{b}\PY{p}{)}
\end{Verbatim}


    'character'

    
    'factor'

    
    Peki neden yaptık bunu? Şimdi yukarıdaki örnekteki \texttt{b}
değişkenini biraz daha inceleyelim. \texttt{levels()} fonksiyonunu
kullanalım.

    \begin{Verbatim}[commandchars=\\\{\}]
{\color{incolor}In [{\color{incolor}30}]:} \PY{n+nf}{levels}\PY{p}{(}\PY{n}{b}\PY{p}{)}
\end{Verbatim}


    \begin{enumerate*}
\item 'E'
\item 'K'
\end{enumerate*}


    
    Gördüğünüz gibi iki adet seviye mevcut. İstersek bunların isimlerini de
değiştirebiliriz.

    \begin{Verbatim}[commandchars=\\\{\}]
{\color{incolor}In [{\color{incolor}31}]:} \PY{n+nf}{levels}\PY{p}{(}\PY{n}{b}\PY{p}{)} \PY{o}{\PYZlt{}\PYZhy{}} \PY{n+nf}{c}\PY{p}{(}\PY{l+s}{\PYZdq{}}\PY{l+s}{Kadın\PYZdq{}}\PY{p}{,}\PY{l+s}{\PYZdq{}}\PY{l+s}{Erkek\PYZdq{}}\PY{p}{)}
         \PY{n}{b}
\end{Verbatim}


    \begin{enumerate*}
\item Erkek
\item Kadın
\item Kadın
\item Erkek
\item Erkek
\end{enumerate*}


    
    Özet yapma kabiliyeti de var bu değişkenlerin. \texttt{summary()}
kullanmamız lazım.

    \begin{Verbatim}[commandchars=\\\{\}]
{\color{incolor}In [{\color{incolor}32}]:} \PY{n+nf}{summary}\PY{p}{(}\PY{n}{b}\PY{p}{)}
\end{Verbatim}


    \begin{description*}
\item[Kadın] 2
\item[Erkek] 3
\end{description*}


    
    \hypertarget{sux131ralux131-faktuxf6r}{%
\subsubsection{Sıralı faktör}\label{sux131ralux131-faktuxf6r}}

    Faktörleri sıralı halde yapabiliriz. \texttt{summary()} fonksiyonu ile
özet yaparsak farkı anlayabiliriz.

    \begin{Verbatim}[commandchars=\\\{\}]
{\color{incolor}In [{\color{incolor}33}]:} \PY{n}{hiz} \PY{o}{\PYZlt{}\PYZhy{}} \PY{n+nf}{c}\PY{p}{(}\PY{l+s}{\PYZdq{}}\PY{l+s}{iki\PYZdq{}}\PY{p}{,}\PY{l+s}{\PYZdq{}}\PY{l+s}{üç\PYZdq{}}\PY{p}{,}\PY{l+s}{\PYZdq{}}\PY{l+s}{bir\PYZdq{}}\PY{p}{,}\PY{l+s}{\PYZdq{}}\PY{l+s}{bir\PYZdq{}}\PY{p}{,}\PY{l+s}{\PYZdq{}}\PY{l+s}{üç\PYZdq{}}\PY{p}{)}
         \PY{n}{a} \PY{o}{\PYZlt{}\PYZhy{}} \PY{n+nf}{factor}\PY{p}{(}\PY{n}{hiz}\PY{p}{,}\PY{n}{ordered}\PY{o}{=}\PY{k+kc}{TRUE}\PY{p}{,}\PY{n}{levels}\PY{o}{=}\PY{n+nf}{c}\PY{p}{(}\PY{l+s}{\PYZdq{}}\PY{l+s}{bir\PYZdq{}}\PY{p}{,}\PY{l+s}{\PYZdq{}}\PY{l+s}{iki\PYZdq{}}\PY{p}{,}\PY{l+s}{\PYZdq{}}\PY{l+s}{üç\PYZdq{}}\PY{p}{)}\PY{p}{)}
         \PY{n+nf}{summary}\PY{p}{(}\PY{n}{a}\PY{p}{)}
\end{Verbatim}


    \begin{description*}
\item[bir] 2
\item[iki] 1
\item[üç] 2
\end{description*}


    
    Sıralı olmasının bir güzelliği de karşılaştırma yapabilmemizdir. Mesela
\texttt{a} nın ilk elemanı ile ikinci elemanını karşılaştıralım.

    \begin{Verbatim}[commandchars=\\\{\}]
{\color{incolor}In [{\color{incolor}34}]:} \PY{n}{a}\PY{n}{[1}\PY{n}{]} \PY{o}{\PYZgt{}} \PY{n}{a} \PY{n}{[2}\PY{n}{]}
         \PY{c+c1}{\PYZsh{} yukarıda sıralı faktörü bir, iki, üç olarak }
         \PY{c+c1}{\PYZsh{} tanımladığımız için ve a[1] = \PYZdq{}iki\PYZdq{} a[2] = \PYZdq{}üç\PYZdq{}}
         \PY{c+c1}{\PYZsh{} olduğu için sonuç FALSE çıkacak}
\end{Verbatim}


    FALSE

    
    \hypertarget{dataframe}{%
\subsection{DataFrame}\label{dataframe}}

    Geldik en önemli veri yapısı olan \texttt{dataframe}'e.
\texttt{dataframe}i bir ``excel'' tablosu olarak düşünebiliriz. Yani
satırlar ve sütunlar mevcut. Her bir satır bir gözlemi her bir sütunda
bir değişkeni belirtiyor.

    \texttt{R} içerisinde gömülü olan örnek \texttt{dataframe} yapıları var.
Bunlardan \texttt{mtcars}'a bakalım

    \begin{Verbatim}[commandchars=\\\{\}]
{\color{incolor}In [{\color{incolor}35}]:} \PY{n}{mtcars}
\end{Verbatim}


    \begin{tabular}{r|lllllllllll}
  & mpg & cyl & disp & hp & drat & wt & qsec & vs & am & gear & carb\\
\hline
	Mazda RX4 & 21.0  & 6     & 160.0 & 110   & 3.90  & 2.620 & 16.46 & 0     & 1     & 4     & 4    \\
	Mazda RX4 Wag & 21.0  & 6     & 160.0 & 110   & 3.90  & 2.875 & 17.02 & 0     & 1     & 4     & 4    \\
	Datsun 710 & 22.8  & 4     & 108.0 &  93   & 3.85  & 2.320 & 18.61 & 1     & 1     & 4     & 1    \\
	Hornet 4 Drive & 21.4  & 6     & 258.0 & 110   & 3.08  & 3.215 & 19.44 & 1     & 0     & 3     & 1    \\
	Hornet Sportabout & 18.7  & 8     & 360.0 & 175   & 3.15  & 3.440 & 17.02 & 0     & 0     & 3     & 2    \\
	Valiant & 18.1  & 6     & 225.0 & 105   & 2.76  & 3.460 & 20.22 & 1     & 0     & 3     & 1    \\
	Duster 360 & 14.3  & 8     & 360.0 & 245   & 3.21  & 3.570 & 15.84 & 0     & 0     & 3     & 4    \\
	Merc 240D & 24.4  & 4     & 146.7 &  62   & 3.69  & 3.190 & 20.00 & 1     & 0     & 4     & 2    \\
	Merc 230 & 22.8  & 4     & 140.8 &  95   & 3.92  & 3.150 & 22.90 & 1     & 0     & 4     & 2    \\
	Merc 280 & 19.2  & 6     & 167.6 & 123   & 3.92  & 3.440 & 18.30 & 1     & 0     & 4     & 4    \\
	Merc 280C & 17.8  & 6     & 167.6 & 123   & 3.92  & 3.440 & 18.90 & 1     & 0     & 4     & 4    \\
	Merc 450SE & 16.4  & 8     & 275.8 & 180   & 3.07  & 4.070 & 17.40 & 0     & 0     & 3     & 3    \\
	Merc 450SL & 17.3  & 8     & 275.8 & 180   & 3.07  & 3.730 & 17.60 & 0     & 0     & 3     & 3    \\
	Merc 450SLC & 15.2  & 8     & 275.8 & 180   & 3.07  & 3.780 & 18.00 & 0     & 0     & 3     & 3    \\
	Cadillac Fleetwood & 10.4  & 8     & 472.0 & 205   & 2.93  & 5.250 & 17.98 & 0     & 0     & 3     & 4    \\
	Lincoln Continental & 10.4  & 8     & 460.0 & 215   & 3.00  & 5.424 & 17.82 & 0     & 0     & 3     & 4    \\
	Chrysler Imperial & 14.7  & 8     & 440.0 & 230   & 3.23  & 5.345 & 17.42 & 0     & 0     & 3     & 4    \\
	Fiat 128 & 32.4  & 4     &  78.7 &  66   & 4.08  & 2.200 & 19.47 & 1     & 1     & 4     & 1    \\
	Honda Civic & 30.4  & 4     &  75.7 &  52   & 4.93  & 1.615 & 18.52 & 1     & 1     & 4     & 2    \\
	Toyota Corolla & 33.9  & 4     &  71.1 &  65   & 4.22  & 1.835 & 19.90 & 1     & 1     & 4     & 1    \\
	Toyota Corona & 21.5  & 4     & 120.1 &  97   & 3.70  & 2.465 & 20.01 & 1     & 0     & 3     & 1    \\
	Dodge Challenger & 15.5  & 8     & 318.0 & 150   & 2.76  & 3.520 & 16.87 & 0     & 0     & 3     & 2    \\
	AMC Javelin & 15.2  & 8     & 304.0 & 150   & 3.15  & 3.435 & 17.30 & 0     & 0     & 3     & 2    \\
	Camaro Z28 & 13.3  & 8     & 350.0 & 245   & 3.73  & 3.840 & 15.41 & 0     & 0     & 3     & 4    \\
	Pontiac Firebird & 19.2  & 8     & 400.0 & 175   & 3.08  & 3.845 & 17.05 & 0     & 0     & 3     & 2    \\
	Fiat X1-9 & 27.3  & 4     &  79.0 &  66   & 4.08  & 1.935 & 18.90 & 1     & 1     & 4     & 1    \\
	Porsche 914-2 & 26.0  & 4     & 120.3 &  91   & 4.43  & 2.140 & 16.70 & 0     & 1     & 5     & 2    \\
	Lotus Europa & 30.4  & 4     &  95.1 & 113   & 3.77  & 1.513 & 16.90 & 1     & 1     & 5     & 2    \\
	Ford Pantera L & 15.8  & 8     & 351.0 & 264   & 4.22  & 3.170 & 14.50 & 0     & 1     & 5     & 4    \\
	Ferrari Dino & 19.7  & 6     & 145.0 & 175   & 3.62  & 2.770 & 15.50 & 0     & 1     & 5     & 6    \\
	Maserati Bora & 15.0  & 8     & 301.0 & 335   & 3.54  & 3.570 & 14.60 & 0     & 1     & 5     & 8    \\
	Volvo 142E & 21.4  & 4     & 121.0 & 109   & 4.11  & 2.780 & 18.60 & 1     & 1     & 4     & 2    \\
\end{tabular}


    
    Bir verinin \texttt{dataframe} yapısında olmasının birçok avantajı var.
Bunları tek tek anlatmak yerine örneklerle gidelim

    \begin{quote}
\textbf{\texttt{head()} ve \texttt{str()} (structure) fonksiyonları}
\end{quote}

    \begin{Verbatim}[commandchars=\\\{\}]
{\color{incolor}In [{\color{incolor}36}]:} \PY{c+c1}{\PYZsh{} ilk örneklerini göstermek istersek}
         \PY{c+c1}{\PYZsh{} n kaç tane örnek gösterileceğini belirtir}
         \PY{n+nf}{head}\PY{p}{(}\PY{n}{mtcars}\PY{p}{,}\PY{n}{n}\PY{o}{=}\PY{l+m}{3}\PY{p}{)}
\end{Verbatim}


    \begin{tabular}{r|lllllllllll}
  & mpg & cyl & disp & hp & drat & wt & qsec & vs & am & gear & carb\\
\hline
	Mazda RX4 & 21.0  & 6     & 160   & 110   & 3.90  & 2.620 & 16.46 & 0     & 1     & 4     & 4    \\
	Mazda RX4 Wag & 21.0  & 6     & 160   & 110   & 3.90  & 2.875 & 17.02 & 0     & 1     & 4     & 4    \\
	Datsun 710 & 22.8  & 4     & 108   &  93   & 3.85  & 2.320 & 18.61 & 1     & 1     & 4     & 1    \\
\end{tabular}


    
    \begin{Verbatim}[commandchars=\\\{\}]
{\color{incolor}In [{\color{incolor}37}]:} \PY{c+c1}{\PYZsh{} Değişkenler hakkında bilgi verir}
         \PY{n+nf}{str}\PY{p}{(}\PY{n}{mtcars}\PY{p}{)}
\end{Verbatim}


    \begin{Verbatim}[commandchars=\\\{\}]
'data.frame':	32 obs. of  11 variables:
 \$ mpg : num  21 21 22.8 21.4 18.7 18.1 14.3 24.4 22.8 19.2 {\ldots}
 \$ cyl : num  6 6 4 6 8 6 8 4 4 6 {\ldots}
 \$ disp: num  160 160 108 258 360 {\ldots}
 \$ hp  : num  110 110 93 110 175 105 245 62 95 123 {\ldots}
 \$ drat: num  3.9 3.9 3.85 3.08 3.15 2.76 3.21 3.69 3.92 3.92 {\ldots}
 \$ wt  : num  2.62 2.88 2.32 3.21 3.44 {\ldots}
 \$ qsec: num  16.5 17 18.6 19.4 17 {\ldots}
 \$ vs  : num  0 0 1 1 0 1 0 1 1 1 {\ldots}
 \$ am  : num  1 1 1 0 0 0 0 0 0 0 {\ldots}
 \$ gear: num  4 4 4 3 3 3 3 4 4 4 {\ldots}
 \$ carb: num  4 4 1 1 2 1 4 2 2 4 {\ldots}

    \end{Verbatim}

    \begin{quote}
\textbf{\texttt{datafame} nasıl oluşturulur?}
\end{quote}

    Hazır veri seti kullanmak kolay diyorsunuz sanırım :) Sıfırdan bir tane
yapalım isterseniz. \texttt{data.frame} ile hazırlıyoruz.

    \begin{Verbatim}[commandchars=\\\{\}]
{\color{incolor}In [{\color{incolor}38}]:} \PY{n}{isim} \PY{o}{\PYZlt{}\PYZhy{}} \PY{n+nf}{c}\PY{p}{(}\PY{l+s}{\PYZdq{}}\PY{l+s}{Ali\PYZdq{}}\PY{p}{,}\PY{l+s}{\PYZdq{}}\PY{l+s}{Ahmet\PYZdq{}}\PY{p}{,}\PY{l+s}{\PYZdq{}}\PY{l+s}{Ayşe\PYZdq{}}\PY{p}{,}\PY{l+s}{\PYZdq{}}\PY{l+s}{Ayla\PYZdq{}}\PY{p}{)}
         \PY{n}{boy} \PY{o}{\PYZlt{}\PYZhy{}} \PY{n+nf}{c}\PY{p}{(}\PY{l+m}{183}\PY{p}{,}\PY{l+m}{172}\PY{p}{,}\PY{l+m}{165}\PY{p}{,}\PY{l+m}{190}\PY{p}{)}
         \PY{n}{cinsiyet} \PY{o}{\PYZlt{}\PYZhy{}} \PY{n+nf}{c}\PY{p}{(}\PY{l+s}{\PYZdq{}}\PY{l+s}{E\PYZdq{}}\PY{p}{,}\PY{l+s}{\PYZdq{}}\PY{l+s}{E\PYZdq{}}\PY{p}{,}\PY{l+s}{\PYZdq{}}\PY{l+s}{K\PYZdq{}}\PY{p}{,}\PY{l+s}{\PYZdq{}}\PY{l+s}{K\PYZdq{}}\PY{p}{)}
         \PY{n}{kilo} \PY{o}{\PYZlt{}\PYZhy{}} \PY{n+nf}{c}\PY{p}{(}\PY{l+m}{90.1}\PY{p}{,}\PY{l+m}{70.4}\PY{p}{,}\PY{l+m}{55.3}\PY{p}{,}\PY{l+m}{65.8}\PY{p}{)}
         \PY{n}{kisiler} \PY{o}{\PYZlt{}\PYZhy{}} \PY{n+nf}{data.frame}\PY{p}{(}\PY{n}{isim}\PY{p}{,}\PY{n}{boy}\PY{p}{,}\PY{n}{cinsiyet}\PY{p}{,}\PY{n}{kilo}\PY{p}{)}
         \PY{n}{kisiler}
         \PY{n+nf}{str}\PY{p}{(}\PY{n}{kisiler}\PY{p}{)}
\end{Verbatim}


    \begin{tabular}{r|llll}
 isim & boy & cinsiyet & kilo\\
\hline
	 Ali   & 183   & E     & 90.1 \\
	 Ahmet & 172   & E     & 70.4 \\
	 Ayşe  & 165   & K     & 55.3 \\
	 Ayla  & 190   & K     & 65.8 \\
\end{tabular}


    
    \begin{Verbatim}[commandchars=\\\{\}]
'data.frame':	4 obs. of  4 variables:
 \$ isim    : Factor w/ 4 levels "Ahmet","Ali",..: 2 1 4 3
 \$ boy     : num  183 172 165 190
 \$ cinsiyet: Factor w/ 2 levels "E","K": 1 1 2 2
 \$ kilo    : num  90.1 70.4 55.3 65.8

    \end{Verbatim}

    Dikkat ederseniz bazı değişkenleri otomatik olarak \texttt{factor}
haline çevrilmiştir.

    \begin{quote}
\textbf{\texttt{dataframe} yapısında matrislerde olduğu gibi kesit alma
yapabiliriz}
\end{quote}

    \begin{Verbatim}[commandchars=\\\{\}]
{\color{incolor}In [{\color{incolor}39}]:} \PY{n}{mtcars}\PY{n}{[1}\PY{p}{,}\PY{n}{]}
\end{Verbatim}


    \begin{tabular}{r|lllllllllll}
  & mpg & cyl & disp & hp & drat & wt & qsec & vs & am & gear & carb\\
\hline
	Mazda RX4 & 21    & 6     & 160   & 110   & 3.9   & 2.62  & 16.46 & 0     & 1     & 4     & 4    \\
\end{tabular}


    
    \begin{Verbatim}[commandchars=\\\{\}]
{\color{incolor}In [{\color{incolor}40}]:} \PY{n}{mtcars}\PY{n}{[}\PY{p}{,}\PY{l+m}{1}\PY{n}{]}
\end{Verbatim}


    \begin{enumerate*}
\item 21
\item 21
\item 22.8
\item 21.4
\item 18.7
\item 18.1
\item 14.3
\item 24.4
\item 22.8
\item 19.2
\item 17.8
\item 16.4
\item 17.3
\item 15.2
\item 10.4
\item 10.4
\item 14.7
\item 32.4
\item 30.4
\item 33.9
\item 21.5
\item 15.5
\item 15.2
\item 13.3
\item 19.2
\item 27.3
\item 26
\item 30.4
\item 15.8
\item 19.7
\item 15
\item 21.4
\end{enumerate*}


    
    \begin{Verbatim}[commandchars=\\\{\}]
{\color{incolor}In [{\color{incolor}41}]:} \PY{n}{mtcars}\PY{n}{[3}\PY{o}{:}\PY{l+m}{5}\PY{p}{,}\PY{l+m}{4}\PY{o}{:}\PY{l+m}{6}\PY{n}{]}
\end{Verbatim}


    \begin{tabular}{r|lll}
  & hp & drat & wt\\
\hline
	Datsun 710 &  93   & 3.85  & 2.320\\
	Hornet 4 Drive & 110   & 3.08  & 3.215\\
	Hornet Sportabout & 175   & 3.15  & 3.440\\
\end{tabular}


    
    \begin{quote}
Doğrudan sütun isimlerini de yazarak erişim veriye gerçekleştirebiliriz.
İkili sütun örneği yapalım.
\end{quote}

    \begin{Verbatim}[commandchars=\\\{\}]
{\color{incolor}In [{\color{incolor}42}]:} \PY{n}{kisiler}\PY{n}{[2}\PY{o}{:}\PY{l+m}{3}\PY{p}{,}\PY{n+nf}{c}\PY{p}{(}\PY{l+s}{\PYZdq{}}\PY{l+s}{kilo\PYZdq{}}\PY{p}{,}\PY{l+s}{\PYZdq{}}\PY{l+s}{boy\PYZdq{}}\PY{p}{)}\PY{n}{]}
\end{Verbatim}


    \begin{tabular}{r|ll}
  & kilo & boy\\
\hline
	2 & 70.4 & 172 \\
	3 & 55.3 & 165 \\
\end{tabular}


    
    \begin{quote}
Genellikle değişkenlere (sütunlara) erişim gerçekleştirmek için
\texttt{\$} işareti kullanılır.
\end{quote}

    \begin{Verbatim}[commandchars=\\\{\}]
{\color{incolor}In [{\color{incolor}43}]:} \PY{n}{kisiler}\PY{o}{\PYZdl{}}\PY{n}{boy}
\end{Verbatim}


    \begin{enumerate*}
\item 183
\item 172
\item 165
\item 190
\end{enumerate*}


    
    \begin{quote}
\textbf{\texttt{dataframe} yapılarında koşullu seçim nasıl yapılır?}
\end{quote}

    Örneğin hazırladığımız ``kisiler'' verisetinde boyu 175 altı olanları
bulalım

    \begin{Verbatim}[commandchars=\\\{\}]
{\color{incolor}In [{\color{incolor}44}]:} \PY{c+c1}{\PYZsh{} bu bir logical seri döndürür}
         \PY{n}{kisiler}\PY{o}{\PYZdl{}}\PY{n}{boy}\PY{o}{\PYZlt{}}\PY{l+m}{175}
         
         \PY{c+c1}{\PYZsh{} bu seriyi satır kısmına veriyoruz sütunların hepsini alıyoruz}
         \PY{c+c1}{\PYZsh{} kisiler[\PYZhy{}\PYZhy{}satırlar\PYZhy{}\PYZhy{} ,\PYZhy{}\PYZhy{}sütunlar\PYZhy{}\PYZhy{}]}
         \PY{n}{kisiler}\PY{n}{[kisiler}\PY{o}{\PYZdl{}}\PY{n}{boy}\PY{o}{\PYZlt{}}\PY{l+m}{175}\PY{p}{,}\PY{n}{]}
\end{Verbatim}


    \begin{enumerate*}
\item FALSE
\item TRUE
\item TRUE
\item FALSE
\end{enumerate*}


    
    \begin{tabular}{r|llll}
  & isim & boy & cinsiyet & kilo\\
\hline
	2 & Ahmet & 172   & E     & 70.4 \\
	3 & Ayşe  & 165   & K     & 55.3 \\
\end{tabular}


    
    Boyu 175 altı ve kilosu 60 üstü olanları bulalım. Burada \texttt{\&}
(ve) operatörü kullanalım.

    \begin{Verbatim}[commandchars=\\\{\}]
{\color{incolor}In [{\color{incolor}45}]:} \PY{n}{kisiler}\PY{n}{[kisiler}\PY{o}{\PYZdl{}}\PY{n}{boy}\PY{o}{\PYZlt{}}\PY{l+m}{175} \PY{o}{\PYZam{}} \PY{n}{kisiler}\PY{o}{\PYZdl{}}\PY{n}{kilo}\PY{o}{\PYZgt{}}\PY{l+m}{60}\PY{p}{,}\PY{n}{]}
\end{Verbatim}


    \begin{tabular}{r|llll}
  & isim & boy & cinsiyet & kilo\\
\hline
	2 & Ahmet & 172   & E     & 70.4 \\
\end{tabular}


    
    \begin{quote}
\textbf{Koşullu seçim yapmak için \texttt{subset} fonksiyonunu da
kullanabiliriz.}
\end{quote}

    \begin{Verbatim}[commandchars=\\\{\}]
{\color{incolor}In [{\color{incolor}46}]:} \PY{n+nf}{subset}\PY{p}{(}\PY{n}{kisiler}\PY{p}{,}\PY{n}{boy}\PY{o}{\PYZlt{}}\PY{l+m}{175}\PY{p}{)}
\end{Verbatim}


    \begin{tabular}{r|llll}
  & isim & boy & cinsiyet & kilo\\
\hline
	2 & Ahmet & 172   & E     & 70.4 \\
	3 & Ayşe  & 165   & K     & 55.3 \\
\end{tabular}


    
    \begin{quote}
\texttt{order()} fonksiyonu sıralama yapmak için kullanılabilir.
\end{quote}

    \begin{Verbatim}[commandchars=\\\{\}]
{\color{incolor}In [{\color{incolor}47}]:} \PY{n+nf}{order}\PY{p}{(}\PY{n}{kisiler}\PY{o}{\PYZdl{}}\PY{n}{boy}\PY{p}{)}
\end{Verbatim}


    \begin{enumerate*}
\item 3
\item 2
\item 1
\item 4
\end{enumerate*}


    
    \begin{Verbatim}[commandchars=\\\{\}]
{\color{incolor}In [{\color{incolor}48}]:} \PY{c+c1}{\PYZsh{} order() sonucunu yine satır kısmına }
         \PY{c+c1}{\PYZsh{} verirsek veri yapısını sıralayabiliriz.}
         \PY{n}{kisiler}\PY{n+nf}{[order}\PY{p}{(}\PY{n}{kisiler}\PY{o}{\PYZdl{}}\PY{n}{boy}\PY{p}{)}\PY{p}{,}\PY{n}{]}
\end{Verbatim}


    \begin{tabular}{r|llll}
  & isim & boy & cinsiyet & kilo\\
\hline
	3 & Ayşe  & 165   & K     & 55.3 \\
	2 & Ahmet & 172   & E     & 70.4 \\
	1 & Ali   & 183   & E     & 90.1 \\
	4 & Ayla  & 190   & K     & 65.8 \\
\end{tabular}


    
    \begin{Verbatim}[commandchars=\\\{\}]
{\color{incolor}In [{\color{incolor}49}]:} \PY{c+c1}{\PYZsh{} Büyükten küçüğe sıralamak istersek}
         \PY{n}{kisiler}\PY{n+nf}{[order}\PY{p}{(}\PY{n}{kisiler}\PY{o}{\PYZdl{}}\PY{n}{kilo}\PY{p}{,}\PY{n}{decreasing} \PY{o}{=} \PY{k+kc}{TRUE}\PY{p}{)}\PY{p}{,}\PY{n}{]}
\end{Verbatim}


    \begin{tabular}{r|llll}
  & isim & boy & cinsiyet & kilo\\
\hline
	1 & Ali   & 183   & E     & 90.1 \\
	2 & Ahmet & 172   & E     & 70.4 \\
	4 & Ayla  & 190   & K     & 65.8 \\
	3 & Ayşe  & 165   & K     & 55.3 \\
\end{tabular}


    
    \hypertarget{lists-listeler}{%
\subsection{Lists (Listeler)}\label{lists-listeler}}

    Listeler genel tanımları itibariyle içine her türlü nesneyi
atabildiğimiz veri yapılarıdır. \texttt{list()} fonksiyonu ile
oluşturulur.

    \begin{Verbatim}[commandchars=\\\{\}]
{\color{incolor}In [{\color{incolor}50}]:} \PY{n}{a} \PY{o}{\PYZlt{}\PYZhy{}} \PY{n+nf}{c}\PY{p}{(}\PY{l+m}{1}\PY{p}{,}\PY{l+m}{2}\PY{p}{,}\PY{l+m}{3}\PY{p}{,}\PY{l+m}{4}\PY{p}{)}
         \PY{n}{b} \PY{o}{\PYZlt{}\PYZhy{}} \PY{n+nf}{matrix}\PY{p}{(}\PY{l+m}{1}\PY{o}{:}\PY{l+m}{9}\PY{p}{,}\PY{n}{nrow}\PY{o}{=}\PY{l+m}{3}\PY{p}{)}
         \PY{n}{c} \PY{o}{\PYZlt{}\PYZhy{}} \PY{k+kc}{TRUE}
         \PY{n}{d} \PY{o}{\PYZlt{}\PYZhy{}} \PY{n}{kisiler} \PY{c+c1}{\PYZsh{} R içine gömülü veri seti}
         \PY{n}{liste} \PY{o}{\PYZlt{}\PYZhy{}} \PY{n+nf}{list}\PY{p}{(}\PY{n}{a}\PY{p}{,}\PY{n}{b}\PY{p}{,}\PY{n}{c}\PY{p}{,}\PY{n}{d}\PY{p}{)}
         \PY{n}{liste}
\end{Verbatim}


    \begin{enumerate}
\item \begin{enumerate*}
\item 1
\item 2
\item 3
\item 4
\end{enumerate*}

\item \begin{tabular}{lll}
	 1 & 4 & 7\\
	 2 & 5 & 8\\
	 3 & 6 & 9\\
\end{tabular}

\item TRUE
\item \begin{tabular}{r|llll}
 isim & boy & cinsiyet & kilo\\
\hline
	 Ali   & 183   & E     & 90.1 \\
	 Ahmet & 172   & E     & 70.4 \\
	 Ayşe  & 165   & K     & 55.3 \\
	 Ayla  & 190   & K     & 65.8 \\
\end{tabular}

\end{enumerate}


    
    Liste yapısına isim de verebiliriz.

    \begin{Verbatim}[commandchars=\\\{\}]
{\color{incolor}In [{\color{incolor}51}]:} \PY{n+nf}{names}\PY{p}{(}\PY{n}{liste}\PY{p}{)} \PY{o}{\PYZlt{}\PYZhy{}} \PY{n+nf}{c}\PY{p}{(}\PY{l+s}{\PYZdq{}}\PY{l+s}{vektör\PYZdq{}}\PY{p}{,}\PY{l+s}{\PYZdq{}}\PY{l+s}{matris\PYZdq{}}\PY{p}{,}\PY{l+s}{\PYZdq{}}\PY{l+s}{logical\PYZdq{}}\PY{p}{,}\PY{l+s}{\PYZdq{}}\PY{l+s}{dataframe\PYZdq{}}\PY{p}{)}
         \PY{n}{liste}
\end{Verbatim}


    \begin{description}
\item[\$vektör] \begin{enumerate*}
\item 1
\item 2
\item 3
\item 4
\end{enumerate*}

\item[\$matris] \begin{tabular}{lll}
	 1 & 4 & 7\\
	 2 & 5 & 8\\
	 3 & 6 & 9\\
\end{tabular}

\item[\$logical] TRUE
\item[\$dataframe] \begin{tabular}{r|llll}
 isim & boy & cinsiyet & kilo\\
\hline
	 Ali   & 183   & E     & 90.1 \\
	 Ahmet & 172   & E     & 70.4 \\
	 Ayşe  & 165   & K     & 55.3 \\
	 Ayla  & 190   & K     & 65.8 \\
\end{tabular}

\end{description}


    
    Listeyi hazırlarken de isim verebiliriz. Burada dikkat edin isimleri
``string'' olarak vermiyoruz.

    \begin{Verbatim}[commandchars=\\\{\}]
{\color{incolor}In [{\color{incolor}52}]:} \PY{n+nf}{list}\PY{p}{(}\PY{n}{vektör}\PY{o}{=}\PY{n}{a}\PY{p}{,}\PY{n}{matris}\PY{o}{=}\PY{n}{b}\PY{p}{,}\PY{n}{logical}\PY{o}{=}\PY{n}{c}\PY{p}{,}\PY{n}{dataframe}\PY{o}{=}\PY{n}{d}\PY{p}{)}
\end{Verbatim}


    \begin{description}
\item[\$vektör] \begin{enumerate*}
\item 1
\item 2
\item 3
\item 4
\end{enumerate*}

\item[\$matris] \begin{tabular}{lll}
	 1 & 4 & 7\\
	 2 & 5 & 8\\
	 3 & 6 & 9\\
\end{tabular}

\item[\$logical] TRUE
\item[\$dataframe] \begin{tabular}{r|llll}
 isim & boy & cinsiyet & kilo\\
\hline
	 Ali   & 183   & E     & 90.1 \\
	 Ahmet & 172   & E     & 70.4 \\
	 Ayşe  & 165   & K     & 55.3 \\
	 Ayla  & 190   & K     & 65.8 \\
\end{tabular}

\end{description}


    
    \begin{quote}
Listenin alt elemanlarını seçerken \texttt{\$} işaretini kullanabiliriz.
Hatta bunu birden fazla kez yapabiliriz.
\end{quote}

    \begin{Verbatim}[commandchars=\\\{\}]
{\color{incolor}In [{\color{incolor}53}]:} \PY{n}{liste}\PY{o}{\PYZdl{}}\PY{n}{dataframe}\PY{o}{\PYZdl{}}\PY{n}{kilo}
\end{Verbatim}


    \begin{enumerate*}
\item 90.1
\item 70.4
\item 55.3
\item 65.8
\end{enumerate*}


    
    \begin{quote}
Listeye bir eleman daha eklemek istersek \texttt{c()} fonksiyonunu
kullanırız.
\end{quote}

    \begin{Verbatim}[commandchars=\\\{\}]
{\color{incolor}In [{\color{incolor}54}]:} \PY{n}{liste} \PY{o}{\PYZlt{}\PYZhy{}} \PY{n+nf}{c}\PY{p}{(}\PY{n}{liste}\PY{p}{,}\PY{n}{ekleme}\PY{o}{=}\PY{l+m}{42}\PY{p}{)}
         \PY{n}{liste}
\end{Verbatim}


    \begin{description}
\item[\$vektör] \begin{enumerate*}
\item 1
\item 2
\item 3
\item 4
\end{enumerate*}

\item[\$matris] \begin{tabular}{lll}
	 1 & 4 & 7\\
	 2 & 5 & 8\\
	 3 & 6 & 9\\
\end{tabular}

\item[\$logical] TRUE
\item[\$dataframe] \begin{tabular}{r|llll}
 isim & boy & cinsiyet & kilo\\
\hline
	 Ali   & 183   & E     & 90.1 \\
	 Ahmet & 172   & E     & 70.4 \\
	 Ayşe  & 165   & K     & 55.3 \\
	 Ayla  & 190   & K     & 65.8 \\
\end{tabular}

\item[\$ekleme] 42
\end{description}


    

    % Add a bibliography block to the postdoc
    
    
    
    \end{document}
